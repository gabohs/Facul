\documentclass{article}
\usepackage{graphicx} % Required for inserting images
\usepackage{tikz}
\usepackage{amsmath}
\usepackage[most]{tcolorbox}
\usepackage{xcolor}
\usepackage[utf8]{inputenc}
\usepackage{amssymb}
\usepackage{emoji}


\definecolor{cblue}{RGB}{30,144,255}

\pagecolor[RGB]{17,21,24} 
\color[RGB]{234,236,240} 

\newtcolorbox{notebox}{
  enhanced,
  frame hidden,                                
  borderline west={2pt}{0pt}{cblue!70!black},  
  colback=white,                              
  boxrule=0pt,
  left=6pt, right=0pt, top=3pt, bottom=3pt,    
}

% -------------------------------------------------

\begin{document}

\begin{titlepage}
    \centering
    \vspace*{\fill}
    {\huge \textbf{Estatística Aula 1:}\par} 
    {\Large \textbf{Anotações}\par} 
    \vspace{0.7cm} 
    {\Large Gabriel H. Schaeffer\par} 
    \vspace{0.5cm} 
    {\large 13/08/2025\par} 
    \vspace*{\fill}
\end{titlepage}

\newpage

\setcounter{section}{-1}
\section{Introdução e Conceitos Básicos}
\vspace{0.5cm}
\subsection{Variável}
É uma característica.

{
\Large
$$
x = \text{idade}
$$
}

\subsection{Dado}
É um valor/valores

{
\Large
$$ 
\lbrace 24, 31, 30 \rbrace
$$
}

\vspace{1cm}
{\noindent \huge \textbf{Tipos de Variáveis}\par}
\vspace{0.5cm}

\section{Quantitativas}
\begin{itemize}
    \item Medem quantidades
    \item São numéricas
\end{itemize}

\vspace{0.2cm}
\subsection{Contínuas}

\begin{itemize}
    \item Medidas
    \item Assumem qualquer valor dentro de um intervalo
\end{itemize}

\vspace{0.5cm}

\begin{center}
\begin{tikzpicture}

    \draw[-] (-0.5,0) -- (5.5,0);

    \foreach \x in {0,1,2,3,4,5}
        \draw (\x,0.1) -- (\x,-0.1) node[below] {\x};

    \foreach \x in {0,0.1,...,5} {
        \draw (\x,0.08) -- (\x,-0.08);
    }

\end{tikzpicture}
\end{center}

\vspace{0.2cm}
\subsection{Discretas}

\begin{itemize}
    \item Contagem
    \item Assumem dados valores dentro de um intervalo
\end{itemize}

\begin{center}
\begin{tikzpicture}[scale=1.2]

    \draw[->] (-0.5,0) -- (6.5,0);

    \foreach \x in {0,1,2,3,4,5,6} {
        \draw (\x,0.15) -- (\x,-0.15) node[below=3pt] {\x};
    }
    
    \foreach \x in {1,2,3,4,5} {
        \filldraw[blue] (\x,0) circle (2pt);
    }
    
    \draw[dashed, red] (1.5,0.4) -- (1.5,-0.4);
    \node[red, above] at (1.5,0.4) {$1.5$};

    \draw[red, thick] (1.35,-0.3) -- (1.65,0.3);

\end{tikzpicture}
\end{center}

\vspace{0.3cm}
\section{Qualitativas}

\begin{itemize}
    \item Categóricas 
\end{itemize}

\vspace{0.2cm}
\subsection{Ordinais}

\begin{itemize}
    \item Possuem ordem/hierarquia entre suas categorias
\end{itemize}

Exemplos:

\[
\begin{array}{l}
\text{Escolaridade} \left\{ 
    \begin{array}{l}
        EF \\
        EM \\
        ES
    \end{array}
\right.
\hspace{2cm}
\text{Escala Likert} \left\{
    \begin{array}{c}
        \text{\emoji{frowning-face}} \\
        \text{\emoji{neutral-face}}  \\
        \text{\emoji{grinning-face}}
    \end{array}
\right.
\end{array}
\]

\vspace{0.2cm}
\subsection{Nominais}

\begin{itemize}
    \item Descrevem algo, mas as categorias não possuem hierarquia entre si
\end{itemize}

\[
\begin{array}{c}
    \text{Religião} \left\{
        \begin{array}{c}
            \text{Católica} \\
            \text{Luterana} \\
            \vdots
        \end{array}
    \right.
    
    \hspace{2cm}
    \text{Estado Civil} \left\{
        \begin{array}{c}
            \text{Solteiro} \\
            \text{Casado} \\
            \vdots
        \end{array}
    \right.
\end{array}
\]

\vspace{0.1cm}
\subsubsection{Dicotômica}

\begin{itemize}
    \item Apenas duas categorias possíveis
\end{itemize}

\[
\text{Dicotômica} \left\{
    \begin{array}{l}
        Sim \\
        Não
    \end{array}
\right.
\]


\end{document}
